% -*- mode: noweb; ess-noweb-default-code-mode: erlang-mode; -*-% ===> this file was generated automatically by noweave --- better not edit it
% \documentclass[twoside]{article}
\documentclass{tufte-handout}
\usepackage{noweb}

\usepackage{color}
% https://commons.wikimedia.org/wiki/File:Erlang_logo.svg
\definecolor{ErlangRed}{HTML}{A90533}

\usepackage{hyperref}
% l2h macro href 2 <a href="#$1">#2</a>
\hypersetup{
  bookmarks=true,
  pdffitwindow=true,
  pdfstartview={FitH},
  pdftitle={My favorite Erlang Program},
  pdfauthor={Joe Armstrong},
  pdfsubject={Erlang and other stuff},
  pdfcreator={Eric Bailey <eric@ericb.me>},
  pdfkeywords={Erlang, universal server, literate programming, noweb},
  colorlinks=true,
  linkcolor=ErlangRed,
  urlcolor=ErlangRed
}

% \usepackage[parfill]{parskip}

\usepackage{amsmath}
\usepackage{amssymb}

\usepackage[outputdir=src/erlang]{minted}

% NOTE: Use Tufte instead of noweb page style.
% \pagestyle{noweb}
% NOTE: Use shift option for wide code.
\noweboptions{smallcode,shortxref,webnumbering,english}

\title{
  My favorite Erlang Program
  \thanks{
    This version of Joe's blog post was translated, edited and annotated by
    \href{https://github.com/yurrriq}{Eric Bailey}
    on \today.\cite{Armstrong2013}
  }
}

% \author{Joe Armstrong\\{\small Translated to {\tt noweb} by Eric Bailey}}
\author{Joe Armstrong}

\date{November 21, 2013}

\newcommand{\stylehook}{\marginpar{\raggedright\sl Style hook}}

\newmintinline[erl]{erlang}{}

\begin{document}
\maketitle
\nwfilename{src/erlang/fav1.nw}\nwbegindocs{1}\nwdocspar

\begin{abstract}
  The other day I got a mail from Dean Galvin from Rowan University. Dean was
  doing an Erlang project so he asked ``What example program would best
  exemplify Erlang''.

  He wanted a small program, that would be suitable for a ten minute talk that
  would best show off the language. I thought for a while ... and quickly wrote
  my favorite program, it's the ``Universal server''.
\end{abstract}

% \tableofcontents
% \newpage

\section{The Universal Server}\label{sec:universal_server}
Normally servers do something. An HTTP server responds to HTTP requests, an FTP
server response to FTP requests and so on. But what about a {\sl Universal
  Server}? Surely we can generalize the idea of a server and make a universal
server, which we can later tell to become a specific server.

Here's my universal server:
\nwenddocs{}\nwbegincode{2}\sublabel{NW3Gelxe-2PSniS-1}\nwmargintag{{\nwtagstyle{}\subpageref{NW3Gelxe-2PSniS-1}}}\moddef{The Universal Server~{\nwtagstyle{}\subpageref{NW3Gelxe-2PSniS-1}}}\endmoddef\nwstartdeflinemarkup\nwusesondefline{\\{NW3Gelxe-3wQJEV-1}}\nwenddeflinemarkup
\nwindexdefn{\nwixident{universal{\_}server}}{universal:unserver}{NW3Gelxe-2PSniS-1}\nwlinkedidentc{universal_server}{NW3Gelxe-2PSniS-1}() ->
    \LA{}Wait for a ``become F'' message and become an F server~{\nwtagstyle{}\subpageref{NW3Gelxe-1bGa3u-1}}\RA{}
    end.
\nwused{\\{NW3Gelxe-3wQJEV-1}}\nwidentdefs{\\{{\nwixident{universal{\_}server}}{universal:unserver}}}\nwendcode{}\nwbegindocs{3}\nwdocspar

\marginnote[\baselineskip]{
  A universal server waits for a \erl{{become, F}} message
  and then becomes an \erl{F} server.
}
\nwenddocs{}\nwbegincode{4}\sublabel{NW3Gelxe-1bGa3u-1}\nwmargintag{{\nwtagstyle{}\subpageref{NW3Gelxe-1bGa3u-1}}}\moddef{Wait for a ``become F'' message and become an F server~{\nwtagstyle{}\subpageref{NW3Gelxe-1bGa3u-1}}}\endmoddef\nwstartdeflinemarkup\nwusesondefline{\\{NW3Gelxe-2PSniS-1}}\nwenddeflinemarkup
receive
    \{become, F\} ->
        F()
\nwused{\\{NW3Gelxe-2PSniS-1}}\nwendcode{}\nwbegindocs{5}\nwdocspar

That was pretty easy. Once I have created a universal server, it just sits and
waits for a \erl{{become, F}} message and then it becomes an \erl{F} server.

\section{The Factorial Server}\label{sec:factorial_server}
A factorial server is a server which waits for an integer and sends back the
factorial of an integer. This is mind-bogglingly simple:
\nwenddocs{}\nwbegincode{6}\sublabel{NW3Gelxe-1ehRvX-1}\nwmargintag{{\nwtagstyle{}\subpageref{NW3Gelxe-1ehRvX-1}}}\moddef{The Factorial Server~{\nwtagstyle{}\subpageref{NW3Gelxe-1ehRvX-1}}}\endmoddef\nwstartdeflinemarkup\nwusesondefline{\\{NW3Gelxe-3wQJEV-1}}\nwenddeflinemarkup
\nwindexdefn{\nwixident{factorial{\_}server}}{factorial:unserver}{NW3Gelxe-1ehRvX-1}\nwlinkedidentc{factorial_server}{NW3Gelxe-1ehRvX-1}() ->
    \LA{}Wait for an integer N and send back factorial(N)~{\nwtagstyle{}\subpageref{NW3Gelxe-1FtCec-1}}\RA{},
            \nwlinkedidentc{factorial_server}{NW3Gelxe-1ehRvX-1}()
    end.

\LA{}The factorial function~{\nwtagstyle{}\subpageref{NW3Gelxe-3ZjObT-1}}\RA{}
\nwused{\\{NW3Gelxe-3wQJEV-1}}\nwidentdefs{\\{{\nwixident{factorial{\_}server}}{factorial:unserver}}}\nwendcode{}\nwbegindocs{7}\nwdocspar
\newpage

\marginnote[\baselineskip]{
  A factorial server simply waits for an integer $n$ and sends back $n!$.
}
\nwenddocs{}\nwbegincode{8}\sublabel{NW3Gelxe-1FtCec-1}\nwmargintag{{\nwtagstyle{}\subpageref{NW3Gelxe-1FtCec-1}}}\moddef{Wait for an integer N and send back factorial(N)~{\nwtagstyle{}\subpageref{NW3Gelxe-1FtCec-1}}}\endmoddef\nwstartdeflinemarkup\nwusesondefline{\\{NW3Gelxe-1ehRvX-1}}\nwenddeflinemarkup
receive
    \{From, N\} ->
        From ! \nwlinkedidentc{factorial}{NW3Gelxe-3ZjObT-1}(N)
\nwused{\\{NW3Gelxe-1ehRvX-1}}\nwidentuses{\\{{\nwixident{factorial}}{factorial}}}\nwindexuse{\nwixident{factorial}}{factorial}{NW3Gelxe-1FtCec-1}\nwendcode{}\nwbegindocs{9}\nwdocspar

\marginnote[\baselineskip]{
  The Erlang definition of \erl{factorial/1} bears a striking resemblance
  to the recurrence relation:
  \begin{align*}
    n! = \begin{cases}
      1 & \text{if } n = 0, \\
      (n-1)!\times n & \text{if } n > 0.
    \end{cases}
  \end{align*}
}
\nwenddocs{}\nwbegincode{10}\sublabel{NW3Gelxe-3ZjObT-1}\nwmargintag{{\nwtagstyle{}\subpageref{NW3Gelxe-3ZjObT-1}}}\moddef{The factorial function~{\nwtagstyle{}\subpageref{NW3Gelxe-3ZjObT-1}}}\endmoddef\nwstartdeflinemarkup\nwusesondefline{\\{NW3Gelxe-1ehRvX-1}}\nwenddeflinemarkup
\nwindexdefn{\nwixident{factorial}}{factorial}{NW3Gelxe-3ZjObT-1}\nwlinkedidentc{factorial}{NW3Gelxe-3ZjObT-1}(0) -> 1;
\nwlinkedidentc{factorial}{NW3Gelxe-3ZjObT-1}(N) -> N * \nwlinkedidentc{factorial}{NW3Gelxe-3ZjObT-1}(N-1).
\nwused{\\{NW3Gelxe-1ehRvX-1}}\nwidentdefs{\\{{\nwixident{factorial}}{factorial}}}\nwendcode{}\nwbegindocs{11}\nwdocspar

Now we're ready to rock and roll...

\section{Putting It All Together}\label{sec:test}
I'll write a little function that creates a
\hyperref[sec:universal_server]{universal server} and sends it a ``become a
\hyperref[sec:factorial_server]{factorial server}'' message. Then I'll send it
an integer, wait for the response, and print the response:
\nwenddocs{}\nwbegincode{12}\sublabel{NW3Gelxe-4Sd2Fj-1}\nwmargintag{{\nwtagstyle{}\subpageref{NW3Gelxe-4Sd2Fj-1}}}\moddef{Putting It All Together~{\nwtagstyle{}\subpageref{NW3Gelxe-4Sd2Fj-1}}}\endmoddef\nwstartdeflinemarkup\nwusesondefline{\\{NW3Gelxe-3wQJEV-1}}\nwenddeflinemarkup
\nwindexdefn{\nwixident{test}}{test}{NW3Gelxe-4Sd2Fj-1}\nwlinkedidentc{test}{NW3Gelxe-4Sd2Fj-1}() ->
    \LA{}Create a universal server~{\nwtagstyle{}\subpageref{NW3Gelxe-1bdq1V-1}}\RA{},
    \LA{}Send it a ``become a factorial server'' message~{\nwtagstyle{}\subpageref{NW3Gelxe-21ZjIj-1}}\RA{},
    \LA{}Send it an integer~{\nwtagstyle{}\subpageref{NW3Gelxe-1C7uCK-1}}\RA{}
    \LA{}Wait for the response and print the response~{\nwtagstyle{}\subpageref{NW3Gelxe-3b0uok-1}}\RA{}
    end.
\nwused{\\{NW3Gelxe-3wQJEV-1}}\nwidentdefs{\\{{\nwixident{test}}{test}}}\nwendcode{}\nwbegindocs{13}\nwdocspar

\marginnote[\baselineskip]{
  \erl{test/0} creates a \hyperref[sec:universal_server]{universal server},
  binding its pid to \erl{Pid};
}
\nwenddocs{}\nwbegincode{14}\sublabel{NW3Gelxe-1bdq1V-1}\nwmargintag{{\nwtagstyle{}\subpageref{NW3Gelxe-1bdq1V-1}}}\moddef{Create a universal server~{\nwtagstyle{}\subpageref{NW3Gelxe-1bdq1V-1}}}\endmoddef\nwstartdeflinemarkup\nwusesondefline{\\{NW3Gelxe-4Sd2Fj-1}}\nwenddeflinemarkup
Pid = spawn(fun \nwlinkedidentc{universal_server}{NW3Gelxe-2PSniS-1}/0)
\nwused{\\{NW3Gelxe-4Sd2Fj-1}}\nwidentuses{\\{{\nwixident{universal{\_}server}}{universal:unserver}}}\nwindexuse{\nwixident{universal{\_}server}}{universal:unserver}{NW3Gelxe-1bdq1V-1}\nwendcode{}\nwbegindocs{15}\nwdocspar

\marginnote[\baselineskip]{
  ... sends \erl{Pid} a ``become a
  \hyperref[sec:factorial_server]{factorial server}'' message;
}
\nwenddocs{}\nwbegincode{16}\sublabel{NW3Gelxe-21ZjIj-1}\nwmargintag{{\nwtagstyle{}\subpageref{NW3Gelxe-21ZjIj-1}}}\moddef{Send it a ``become a factorial server'' message~{\nwtagstyle{}\subpageref{NW3Gelxe-21ZjIj-1}}}\endmoddef\nwstartdeflinemarkup\nwusesondefline{\\{NW3Gelxe-4Sd2Fj-1}}\nwenddeflinemarkup
Pid ! \{become, fun \nwlinkedidentc{factorial_server}{NW3Gelxe-1ehRvX-1}/0\}
\nwused{\\{NW3Gelxe-4Sd2Fj-1}}\nwidentuses{\\{{\nwixident{factorial{\_}server}}{factorial:unserver}}}\nwindexuse{\nwixident{factorial{\_}server}}{factorial:unserver}{NW3Gelxe-21ZjIj-1}\nwendcode{}\nwbegindocs{17}\nwdocspar

\marginnote[\baselineskip]{
  ... sends \erl{Pid} $50$, asking the
  \hyperref[sec:factorial_server]{factorial server} to compute and respond with
  the value of $50!$;
}
\nwenddocs{}\nwbegincode{18}\sublabel{NW3Gelxe-1C7uCK-1}\nwmargintag{{\nwtagstyle{}\subpageref{NW3Gelxe-1C7uCK-1}}}\moddef{Send it an integer~{\nwtagstyle{}\subpageref{NW3Gelxe-1C7uCK-1}}}\endmoddef\nwstartdeflinemarkup\nwusesondefline{\\{NW3Gelxe-4Sd2Fj-1}}\nwenddeflinemarkup
Pid ! \{self(), 50\},
\nwused{\\{NW3Gelxe-4Sd2Fj-1}}\nwendcode{}\nwbegindocs{19}\nwdocspar

\marginnote{... waits for the response and prints the response.}
\nwenddocs{}\nwbegincode{20}\sublabel{NW3Gelxe-3b0uok-1}\nwmargintag{{\nwtagstyle{}\subpageref{NW3Gelxe-3b0uok-1}}}\moddef{Wait for the response and print the response~{\nwtagstyle{}\subpageref{NW3Gelxe-3b0uok-1}}}\endmoddef\nwstartdeflinemarkup\nwusesondefline{\\{NW3Gelxe-4Sd2Fj-1}}\nwenddeflinemarkup
receive
    X ->
        io:format("~w~n", [X])
\nwused{\\{NW3Gelxe-4Sd2Fj-1}}\nwendcode{}\nwbegindocs{21}\nwdocspar

%% TODO: Add a link to fav1.erl
All these functions belong to the module
\href{https://github.com/yurrriq/learning-noweb/blob/master/src/erlang/fav1.erl}{\erl{fav1}}:
\marginnote[\baselineskip]{N.B. \erl{fav1} exports only \erl{test/0}.}
\nwenddocs{}\nwbegincode{22}\sublabel{NW3Gelxe-3wQJEV-1}\nwmargintag{{\nwtagstyle{}\subpageref{NW3Gelxe-3wQJEV-1}}}\moddef{fav1~{\nwtagstyle{}\subpageref{NW3Gelxe-3wQJEV-1}}}\endmoddef\nwstartdeflinemarkup\nwenddeflinemarkup
-module(fav1).
-export([\nwlinkedidentc{test}{NW3Gelxe-4Sd2Fj-1}/0]).

\LA{}Putting It All Together~{\nwtagstyle{}\subpageref{NW3Gelxe-4Sd2Fj-1}}\RA{}

\LA{}The Universal Server~{\nwtagstyle{}\subpageref{NW3Gelxe-2PSniS-1}}\RA{}

\LA{}The Factorial Server~{\nwtagstyle{}\subpageref{NW3Gelxe-1ehRvX-1}}\RA{}
\nwnotused{fav1}\nwidentuses{\\{{\nwixident{test}}{test}}}\nwindexuse{\nwixident{test}}{test}{NW3Gelxe-3wQJEV-1}\nwendcode{}

\nwixlogsorted{c}{{Create a universal server}{NW3Gelxe-1bdq1V-1}{\nwixu{NW3Gelxe-4Sd2Fj-1}\nwixd{NW3Gelxe-1bdq1V-1}}}%
\nwixlogsorted{c}{{fav1}{NW3Gelxe-3wQJEV-1}{\nwixd{NW3Gelxe-3wQJEV-1}}}%
\nwixlogsorted{c}{{Putting It All Together}{NW3Gelxe-4Sd2Fj-1}{\nwixd{NW3Gelxe-4Sd2Fj-1}\nwixu{NW3Gelxe-3wQJEV-1}}}%
\nwixlogsorted{c}{{Send it a ``become a factorial server'' message}{NW3Gelxe-21ZjIj-1}{\nwixu{NW3Gelxe-4Sd2Fj-1}\nwixd{NW3Gelxe-21ZjIj-1}}}%
\nwixlogsorted{c}{{Send it an integer}{NW3Gelxe-1C7uCK-1}{\nwixu{NW3Gelxe-4Sd2Fj-1}\nwixd{NW3Gelxe-1C7uCK-1}}}%
\nwixlogsorted{c}{{The factorial function}{NW3Gelxe-3ZjObT-1}{\nwixu{NW3Gelxe-1ehRvX-1}\nwixd{NW3Gelxe-3ZjObT-1}}}%
\nwixlogsorted{c}{{The Factorial Server}{NW3Gelxe-1ehRvX-1}{\nwixd{NW3Gelxe-1ehRvX-1}\nwixu{NW3Gelxe-3wQJEV-1}}}%
\nwixlogsorted{c}{{The Universal Server}{NW3Gelxe-2PSniS-1}{\nwixd{NW3Gelxe-2PSniS-1}\nwixu{NW3Gelxe-3wQJEV-1}}}%
\nwixlogsorted{c}{{Wait for a ``become F'' message and become an F server}{NW3Gelxe-1bGa3u-1}{\nwixu{NW3Gelxe-2PSniS-1}\nwixd{NW3Gelxe-1bGa3u-1}}}%
\nwixlogsorted{c}{{Wait for an integer N and send back factorial(N)}{NW3Gelxe-1FtCec-1}{\nwixu{NW3Gelxe-1ehRvX-1}\nwixd{NW3Gelxe-1FtCec-1}}}%
\nwixlogsorted{c}{{Wait for the response and print the response}{NW3Gelxe-3b0uok-1}{\nwixu{NW3Gelxe-4Sd2Fj-1}\nwixd{NW3Gelxe-3b0uok-1}}}%
\nwixlogsorted{i}{{\nwixident{factorial}}{factorial}}%
\nwixlogsorted{i}{{\nwixident{factorial{\_}server}}{factorial:unserver}}%
\nwixlogsorted{i}{{\nwixident{test}}{test}}%
\nwixlogsorted{i}{{\nwixident{universal{\_}server}}{universal:unserver}}%
\nwbegindocs{23}$\square$

Now all we have to do is fire up an Erlang shell and run the test program:
\begin{minted}[]{shell-session}
$ erl
\end{minted}
\begin{minted}[]{erl}
1> c("src/erlang/fav1.erl").
{ok, fav1}
2> fav1:test().
30414093201713378043612608166064768844377641568960512000000000000
ok
\end{minted}

Alternatively, we can simply compile and evaluate it from the command line:
\begin{minted}[]{shell-session}
$ erlc src/erlang/fav1.erl
$ erl -noshell -eval "fav1:test()" -s init stop
30414093201713378043612608166064768844377641568960512000000000000
\end{minted}
\newpage

\section{Aside}
A few years ago when I was at SICS I had access to
\href{http://www.planet-lab.org}{Planet Lab}. Planet Lab is a research network
of 9000 computers. Joining Planet Lab is easy, all you have to do is buy a
standard PC, connect it to the network and donate its use to the Planet Lab
organization. Having donated your machine to the network, you can use all the
other machines.

Planet Lab is a real-world test-bed for distributed applications. It currently
has 1171 nodes at 562 sites.

What was I going to do with Planet Lab? I didn't have a clue. What I ended up
doing was making some scripts to install empty universal Erlang servers on all
the Planet Lab machines (pretty much like the code in this article) - then I set
up a gossip algorithm to flood the network with \erl{become} messages. Then I
had an empty network that in a few seconds would become anything I wanted it to.

About a year later I had to write a paper. One of the disadvantages of being a
researcher is that in order to get money you have to write a paper about
something or other. The paper is never about what currently interests you at the
moment, but rather what the project that financed your research expects to read
about.

Well I had my gossip network setup on Planet Lab and I could tell it to become
anything, so I told it to become a content distribution network and used a
gossip algorithm to make copies of the same file on all machines on the network
and wrote a paper about it and everybody was happy.

\textcopyright\ 2014-2016 Joe Armstrong - All Rights Reserved.
\newpage

\section{Chunks}
\nowebchunks

\section{Index}
\nowebindex

\bibliography{fav1}
\bibliographystyle{plainnat}

\end{document}
\nwenddocs{}
