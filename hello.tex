% -*- mode: noweb; ess-noweb-default-code-mode: idris-mode; -*-% ===> this file was generated automatically by noweave --- better not edit it
\documentclass[twoside]{article}
\usepackage{noweb}
\pagestyle{noweb}
\noweboptions{longchunks,smallcode}
\title{Hello, {\tt noweb}!\\
{\small An exploration of literate Idris programming via {\tt noweb}.}}
\author{Eric Bailey}
\newcommand{\stylehook}{\marginpar{\raggedright\sl Style hook}}
\begin{document}
\maketitle
\nwfilename{hello.nw}\nwbegindocs{1}\nwdocspar

\begin{abstract}
  What follows is an attempt at using {\tt noweb} to write a literate program in
  Idris. While Idris provides some literate programming support of its own, it's
  rather basic (like Literate Haskell), and doesn't allow users to present code
  chunks out of order or do any sort of cross-referencing.
\end{abstract}

\tableofcontents
\newpage

\section{{\tt hello.idr} Outline}
{\tt hello.idr} is rather simple. It consists of a module declaration with a
type signature and definition for the {\tt main} function.

\nwenddocs{}\nwbegincode{2}\sublabel{NW1qNmhI-1p0Y9w-1}\nwmargintag{{\nwtagstyle{}\subpageref{NW1qNmhI-1p0Y9w-1}}}\moddef{*~{\nwtagstyle{}\subpageref{NW1qNmhI-1p0Y9w-1}}}\endmoddef\nwstartdeflinemarkup\nwenddeflinemarkup
\LA{}Module declaration~{\nwtagstyle{}\subpageref{NW1qNmhI-1pw4rR-1}}\RA{}

\LA{}main type signature~{\nwtagstyle{}\subpageref{NW1qNmhI-4Fe3ro-1}}\RA{}
\LA{}main definition~{\nwtagstyle{}\subpageref{NW1qNmhI-3i7yrG-1}}\RA{}
\nwnotused{*}\nwendcode{}\nwbegindocs{3}\nwdocspar

\section{Module Declaration}
Declare the module {\tt Main}, including a docstring, which is another
{\tt noweb} chunk.

\nwenddocs{}\nwbegincode{4}\sublabel{NW1qNmhI-1pw4rR-1}\nwmargintag{{\nwtagstyle{}\subpageref{NW1qNmhI-1pw4rR-1}}}\moddef{Module declaration~{\nwtagstyle{}\subpageref{NW1qNmhI-1pw4rR-1}}}\endmoddef\nwstartdeflinemarkup\nwusesondefline{\\{NW1qNmhI-1p0Y9w-1}}\nwenddeflinemarkup
||| \LA{}Hello message~{\nwtagstyle{}\subpageref{NW1qNmhI-4FK8wZ-1}}\RA{}
module Main
\nwused{\\{NW1qNmhI-1p0Y9w-1}}\nwendcode{}\nwbegindocs{5}\nwdocspar

The docstring above consists of the following message, which we'll also print
later, using a {\tt noweb} reference.
\nwenddocs{}\nwbegincode{6}\sublabel{NW1qNmhI-4FK8wZ-1}\nwmargintag{{\nwtagstyle{}\subpageref{NW1qNmhI-4FK8wZ-1}}}\moddef{Hello message~{\nwtagstyle{}\subpageref{NW1qNmhI-4FK8wZ-1}}}\endmoddef\nwstartdeflinemarkup\nwusesondefline{\\{NW1qNmhI-1pw4rR-1}\\{NW1qNmhI-3i7yrG-1}}\nwenddeflinemarkup
Hello, noweb!
\nwused{\\{NW1qNmhI-1pw4rR-1}\\{NW1qNmhI-3i7yrG-1}}\nwendcode{}\nwbegindocs{7}\nwdocspar

\section{The {\tt main} Function}
{\tt main} is an {\tt IO} action that doesn't return any value, i.e.
\nwenddocs{}\nwbegincode{8}\sublabel{NW1qNmhI-4Fe3ro-1}\nwmargintag{{\nwtagstyle{}\subpageref{NW1qNmhI-4Fe3ro-1}}}\moddef{main type signature~{\nwtagstyle{}\subpageref{NW1qNmhI-4Fe3ro-1}}}\endmoddef\nwstartdeflinemarkup\nwusesondefline{\\{NW1qNmhI-1p0Y9w-1}}\nwenddeflinemarkup
main : IO ()
\nwused{\\{NW1qNmhI-1p0Y9w-1}}\nwendcode{}\nwbegindocs{9}\nwdocspar

Output the message to stdout with a trailing newline.
\nwenddocs{}\nwbegincode{10}\sublabel{NW1qNmhI-3i7yrG-1}\nwmargintag{{\nwtagstyle{}\subpageref{NW1qNmhI-3i7yrG-1}}}\moddef{main definition~{\nwtagstyle{}\subpageref{NW1qNmhI-3i7yrG-1}}}\endmoddef\nwstartdeflinemarkup\nwusesondefline{\\{NW1qNmhI-1p0Y9w-1}}\nwenddeflinemarkup
main = putStrLn "\LA{}Hello message~{\nwtagstyle{}\subpageref{NW1qNmhI-4FK8wZ-1}}\RA{}"
\nwused{\\{NW1qNmhI-1p0Y9w-1}}\nwendcode{}

\nwixlogsorted{c}{{*}{NW1qNmhI-1p0Y9w-1}{\nwixd{NW1qNmhI-1p0Y9w-1}}}%
\nwixlogsorted{c}{{Hello message}{NW1qNmhI-4FK8wZ-1}{\nwixu{NW1qNmhI-1pw4rR-1}\nwixd{NW1qNmhI-4FK8wZ-1}\nwixu{NW1qNmhI-3i7yrG-1}}}%
\nwixlogsorted{c}{{main definition}{NW1qNmhI-3i7yrG-1}{\nwixu{NW1qNmhI-1p0Y9w-1}\nwixd{NW1qNmhI-3i7yrG-1}}}%
\nwixlogsorted{c}{{main type signature}{NW1qNmhI-4Fe3ro-1}{\nwixu{NW1qNmhI-1p0Y9w-1}\nwixd{NW1qNmhI-4Fe3ro-1}}}%
\nwixlogsorted{c}{{Module declaration}{NW1qNmhI-1pw4rR-1}{\nwixu{NW1qNmhI-1p0Y9w-1}\nwixd{NW1qNmhI-1pw4rR-1}}}%
\nwbegindocs{11}\nwdocspar

\section{Chunks} \nowebchunks
% \twocolumn[\section{Index}]
% \nowebindex*
\end{document}
\nwenddocs{}
